\documentclass{sigchi-ext}
% Please be sure that you have the dependencies (i.e., additional
% LaTeX packages) to compile this example.
\usepackage[T1]{fontenc}
\usepackage{textcomp}
\usepackage[scaled=.92]{helvet} % for proper fonts
\usepackage{graphicx} % for EPS use the graphics package instead
\usepackage{balance}  % for useful for balancing the last columns
\usepackage{booktabs} % for pretty table rules
\usepackage{ccicons}  % for Creative Commons citation icons
\usepackage{ragged2e} % for tighter hyphenation

% Some optional stuff you might like/need.
% \usepackage{marginnote} 
% \usepackage[shortlabels]{enumitem}
% \usepackage{paralist}
% \usepackage[utf8]{inputenc} % for a UTF8 editor only

%% EXAMPLE BEGIN -- HOW TO OVERRIDE THE DEFAULT COPYRIGHT STRIP --
% \copyrightinfo{Permission to make digital or hard copies of all or
% part of this work for personal or classroom use is granted without
% fee provided that copies are not made or distributed for profit or
% commercial advantage and that copies bear this notice and the full
% citation on the first page. Copyrights for components of this work
% owned by others than ACM must be honored. Abstracting with credit is
% permitted. To copy otherwise, or republish, to post on servers or to
% redistribute to lists, requires prior specific permission and/or a
% fee. Request permissions from permissions@acm.org.\\
% {\emph{CHI'14}}, April 26--May 1, 2014, Toronto, Canada. \\
% Copyright \copyright~2014 ACM ISBN/14/04...\$15.00. \\
% DOI string from ACM form confirmation}
%% EXAMPLE END

% Paper metadata (use plain text, for PDF inclusion and later
% re-using, if desired).  Use \emtpyauthor when submitting for review
% so you remain anonymous.
\def\plaintitle{ArtRate: Using Bio Signals To Influence Music} \def\plainauthor{First Author, Second Author, Third Author,
  Fourth Author}
\def\emptyauthor{}
\def\plainkeywords{Bio Signals; Heart Rate; Respiration Rate; Music}
\def\plaingeneralterms{Documentation, Standardization}

\title{ArtRate: Using Bio Signals To Influence Music}

\numberofauthors{4}
% Notice how author names are alternately typesetted to appear ordered
% in 2-column format; i.e., the first 4 autors on the first column and
% the other 4 auhors on the second column. Actually, it's up to you to
% strictly adhere to this author notation.
\author{%
  \alignauthor{%
    \textbf{Tamara Bernhard}\\
    \affaddr{University of Freiburg} \\
    \email{bernhard@informatik.uni-freiburg.de} }\alignauthor{%
    \textbf{Armin Lauble}\\
    \affaddr{University of Freiburg}\\
    \email{armin.lauble@neptun.uni-freiburg.de} } \vfil \alignauthor{%
    \textbf{Peter Härlen}\\
    \affaddr{University of Freiburg}\\
    \email{peter.haerlen@students.uni-freiburg.de} }\alignauthor{%
    \textbf{Maximilian Rohland}\\
    \affaddr{University of Freiburg}\\
    \email{wcs@rohlandm.de} } \vfil \alignauthor{%
     } }

% Make sure hyperref comes last of your loaded packages, to give it a
% fighting chance of not being over-written, since its job is to
% redefine many LaTeX commands.
\definecolor{linkColor}{RGB}{6,125,233}
\hypersetup{%
  pdftitle={\plaintitle},
%  pdfauthor={\plainauthor},
  pdfauthor={\emptyauthor},
  pdfkeywords={\plainkeywords},
  bookmarksnumbered,
  pdfstartview={FitH},
  colorlinks,
  citecolor=black,
  filecolor=black,
  linkcolor=black,
  urlcolor=linkColor,
  breaklinks=true,
}

% \reversemarginpar%

\begin{document}

%% For the camera ready, use the commands provided by the ACM in the Permission Release Form.
%\CopyrightYear{2007}
%\setcopyright{rightsretained}
%\conferenceinfo{WOODSTOCK}{'97 El Paso, Texas USA}
%\isbn{0-12345-67-8/90/01}
%\doi{http://dx.doi.org/10.1145/2858036.2858119}
%% Then override the default copyright message with the \acmcopyright command.
%\copyrightinfo{\acmcopyright}

\maketitle

% Uncomment to disable hyphenation (not recommended)
% https://twitter.com/anjirokhan/status/546046683331973120
\RaggedRight{} 

% Do not change the page size or page settings.
\begin{abstract}
  A human body continuously produces various bio signals. The most commonly
  mentioned is the heart rate, but there are other measurable values, e.g.
  the rate of respiration. These signals can be measured from musicians and
  there listeners to influence sound, light or other effects utilized during a
  performance. During this course, we built a hard- and software setup to measure
  heart and respiration rate, post process it and distribute the extracted information
  via open standards to third party applications.
\end{abstract}

\keywords{\plainkeywords}

\category{H.5.m}{Information interfaces and presentation (e.g.,
  HCI)}{Miscellaneous}\category{See}{\url{http://acm.org/about/class/1998/}}{for
  full list of ACM classifiers. This section is required.}
% TODO: write

% TODO: write
\section{Introduction}
\begin{itemize}
  \item effect of music on heart\cite{dousty,shin,tsuroka,inesta2008heart}
  \item read heart and respiration rate of artist and/or listener
  \item generate effects based on bio signals
  \item try to control elements using by deliberately changing the respiration rate
\end{itemize}
\section{Technical Background}

\subsection{Heart Rate}

The heart rate is mainly measured using the ``photopletysmogram'' technology (PPG).
It uses light emitting diodes to illuminate a part of skin, e.g. on the wrist, upper arm
or finger tip and measures the light absorption of the skin with a photo diode. The frequency
between peaks of the absorption translates to the heart rate. %TODO: Sensorpeople: explain & specify

If it is compared to other methods it has the big advantage that it is very easy and fast
to use, there is no necessity for chest straps or multiple electrodes. It is widely used
in modern fitness and lifestyle devices like sports trackers and smart watches.

However, it is possible to use our infrastructure with sensors using other technologies
as long as they support the Bluetooth Low Energy standard.

\subsection{Respiration Rate}

To measure the respiration rate we use a three dimensional accelerometer. The recorded
acceleration values are analysed using a setup designed by Maximilian Kurscheidt for
the surveillance of hospital patients\cite{kurscheidt2016open}.

At first the data is filtered via a bandpass filter, removing frequencies lower and higher
than a healthy persons normal respiration rate bandwidth. Afterwards, the signal goes through
Fast Fourier Transformation and the spectral powers are calculated. From this the frequency is
derived. %TODO: ESE-People: explain & specify

\section{Equipment \& Setup}

The main component of our setup is our OSC\footnote{\url{https://en.wikipedia.org/wiki/Open_Sound_Control}}
server\footnote{\url{https://github.com/ARTrate/Server}}. It is written in Python3 and uses the numpy and
scipy libraries for signal processing.
We chose OSC because it is very easy to implement while being fairly versatile in regard to
the payloads it supports. The server can serve in two different ways: On the one hand it can
focus on the signal processing only while forwarding the processed data via OSC to arbitrary
effect engines, e.g. Ableton. If no such third party software is available the server supports
a variation of effects to demonstrate the capabilities of the sensor setups: There are two sound
effect settings based on the heart rate, a visual representation of the history of the heart rate
and a color change of the history representation based on the respiration rate. %TODO: Tamara: specify

The data is gathered by two sensor setups, one based on commercial products which an artist or
listener might already have at hand and another custom build solution. Only the latter supports
the recording of the respiration rate.

\subsection{commercial}

The commercial setup consists of any heart rate sensor supporting the Bluetooth Low Energy
standard. These sensors are connected to an Android phone running our app\footnote{\url{https://github.com/ARTrate/ARTrate-Android}}.
The app forwards the data from the sensor via OSC to a specified network address. Because
the data is send via UDP the app is not aware of possible packet drops or a wrongly set address.

\subsection{custom}
\begin{itemize}
  \item ESP32
  \item PPG Sensor
  \item Acceleration Sensor
  \item standard powerbank
  \item OSC over WiFi
  \item Signal postprocessing on server
  \item 3D-Printed case
\end{itemize}

\section{Results}

\begin{itemize}
  \item heart rate of artist slows down when starts playing
  \item heart rate of listener speeds up on suspension building phases and drops on ``relaxation''
  \item can be used for effects
  \item respiration rate works technically, but the bandwidth is very low, so it is hard to get nice effects
  \item even if it is possible to influence the respiration rate it is not feasible as a device to
        control the music
\end{itemize}

\balance{} 

\bibliographystyle{SIGCHI-Reference-Format}
\bibliography{references}

\end{document}

%%% Local Variables:
%%% mode: latex
%%% TeX-master: t
%%% End:
