\documentclass{sigchi-ext}
% Please be sure that you have the dependencies (i.e., additional
% LaTeX packages) to compile this example.
\usepackage[T1]{fontenc}
\usepackage{textcomp}
\usepackage[scaled=.92]{helvet} % for proper fonts
\usepackage{graphicx} % for EPS use the graphics package instead
\usepackage{balance}  % for useful for balancing the last columns
\usepackage{booktabs} % for pretty table rules
\usepackage{ccicons}  % for Creative Commons citation icons
\usepackage{ragged2e} % for tighter hyphenation

% Some optional stuff you might like/need.
% \usepackage{marginnote} 
% \usepackage[shortlabels]{enumitem}
% \usepackage{paralist}
% \usepackage[utf8]{inputenc} % for a UTF8 editor only

%% EXAMPLE BEGIN -- HOW TO OVERRIDE THE DEFAULT COPYRIGHT STRIP --
% \copyrightinfo{Permission to make digital or hard copies of all or
% part of this work for personal or classroom use is granted without
% fee provided that copies are not made or distributed for profit or
% commercial advantage and that copies bear this notice and the full
% citation on the first page. Copyrights for components of this work
% owned by others than ACM must be honored. Abstracting with credit is
% permitted. To copy otherwise, or republish, to post on servers or to
% redistribute to lists, requires prior specific permission and/or a
% fee. Request permissions from permissions@acm.org.\\
% {\emph{CHI'14}}, April 26--May 1, 2014, Toronto, Canada. \\
% Copyright \copyright~2014 ACM ISBN/14/04...\$15.00. \\
% DOI string from ACM form confirmation}
%% EXAMPLE END

% Paper metadata (use plain text, for PDF inclusion and later
% re-using, if desired).  Use \emtpyauthor when submitting for review
% so you remain anonymous.
\def\plaintitle{ArtRate: Using Bio Signals To Influence Music} \def\plainauthor{First Author, Second Author, Third Author,
  Fourth Author}
\def\emptyauthor{}
\def\plainkeywords{Bio Signals; Heart Rate; Respiration Rate; Music}
\def\plaingeneralterms{Documentation, Standardization}

\title{ArtRate: Using Bio Signals To Influence Music}

\numberofauthors{4}
% Notice how author names are alternately typesetted to appear ordered
% in 2-column format; i.e., the first 4 autors on the first column and
% the other 4 auhors on the second column. Actually, it's up to you to
% strictly adhere to this author notation.
\author{%
  \alignauthor{%
    \textbf{Tamara Bernhard}\\
    \affaddr{University of Freiburg} \\
    \email{bernhard@informatik.uni-freiburg.de} }\alignauthor{%
    \textbf{Armin Lauble}\\
    \affaddr{University of Freiburg}\\
    \email{armin.lauble@neptun.uni-freiburg.de} } \vfil \alignauthor{%
    \textbf{Peter Härlen}\\
    \affaddr{University of Freiburg}\\
    \email{peter.haerlen@students.uni-freiburg.de} }\alignauthor{%
    \textbf{Maximilian Rohland}\\
    \affaddr{University of Freiburg}\\
    \email{wcs@rohlandm.de} } \vfil \alignauthor{%
     } }

% Make sure hyperref comes last of your loaded packages, to give it a
% fighting chance of not being over-written, since its job is to
% redefine many LaTeX commands.
\definecolor{linkColor}{RGB}{6,125,233}
\hypersetup{%
  pdftitle={\plaintitle},
%  pdfauthor={\plainauthor},
  pdfauthor={\emptyauthor},
  pdfkeywords={\plainkeywords},
  bookmarksnumbered,
  pdfstartview={FitH},
  colorlinks,
  citecolor=black,
  filecolor=black,
  linkcolor=black,
  urlcolor=linkColor,
  breaklinks=true,
}

% \reversemarginpar%

\begin{document}

%% For the camera ready, use the commands provided by the ACM in the Permission Release Form.
%\CopyrightYear{2007}
%\setcopyright{rightsretained}
%\conferenceinfo{WOODSTOCK}{'97 El Paso, Texas USA}
%\isbn{0-12345-67-8/90/01}
%\doi{http://dx.doi.org/10.1145/2858036.2858119}
%% Then override the default copyright message with the \acmcopyright command.
%\copyrightinfo{\acmcopyright}

\maketitle

% Uncomment to disable hyphenation (not recommended)
% https://twitter.com/anjirokhan/status/546046683331973120
\RaggedRight{} 

% Do not change the page size or page settings.
\begin{abstract}
  A human body continuoulsy produces various bio signals. The most commonly
  mentioned is the heart rate, but there are other measurable values, e.g.
  the rate of respiration. These signals can be measured from musicians and
  there listeners to influence sound, light or other effects utilized during a
  performance. During this course, we built a hard- and software setup to measure
  heart and respiration rate, postprocess it and distribute the extracted information
  via open standards to third party applications.
\end{abstract}

\keywords{\plainkeywords}

\category{H.5.m}{Information interfaces and presentation (e.g.,
  HCI)}{Miscellaneous}\category{See}{\url{http://acm.org/about/class/1998/}}{for
  full list of ACM classifiers. This section is required.}
% TODO: write

% TODO: write
\section{Introduction}

\section{Technical Background Stuff}

\subsection{Heart Rate}

\subsection{Respiration Rate}
Based on Kurscheidt\cite{kurscheidt2016open}.

\section{Equipment}

\section{Results}


% \marginpar{%
  % \vspace{-45pt} \fbox{%
    % \begin{minipage}{0.925\marginparwidth}
      % \textbf{Good Utilization of the Side Bar} \\
      % \vspace{1pc} \textbf{Preparation:} Do not change the margin
      % dimensions and do not flow the margin text to the
      % next page. \\
      % \vspace{1pc} \textbf{Materials:} The margin box must not intrude
      % or overflow into the header or the footer, or the gutter space
      % between the margin paragraph and the main left column. The text
      % in this text box should remain the same size as the body
      % text. Use the \texttt{{\textbackslash}vspace{}} command to set
      % the margin
      % note's position. \\
      % \vspace{1pc} \textbf{Images \& Figures:} Practically anything
      % can be put in the margin if it fits. Use the
      % \texttt{{\textbackslash}marginparwidth} constant to set the
      % width of the figure, table, minipage, or whatever you are trying
      % to fit in this skinny space.
    % \end{minipage}}\label{sec:sidebar} }

\balance{} 

\bibliographystyle{SIGCHI-Reference-Format}
\bibliography{references}

\end{document}

%%% Local Variables:
%%% mode: latex
%%% TeX-master: t
%%% End:
